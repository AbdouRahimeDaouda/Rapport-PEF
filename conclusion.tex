\chapter*{Conclusion générale}
\addcontentsline{toc}{chapter}{Conclusion générale}
\markboth{Conclusion générale}{}
Adopter l'approche des micro-services n'est pas un sujet prospectif, mais une réalité pour les entreprises, d'autant plus que des géants du web comme Netflix et Amazon le trouvent efficace et fiable. En effet, les entreprises devraient accorder dorénavant une grande importance au choix de l'architecture de leur système en vue d'améliorer la performance de leurs activités.\\ \\
Le projet de refonte architecturale de l'ERP "BYBLOS" réalisé par \textbf{Talan Tunisie}, particulièrement le module de gestion des candidats, est un pas en avant vers une agilité de bout en bout. L'objectif final, n'est qu'une pure volonté stratégique, vise à faire évoluer l'organisation de l'entreprise, mais aussi à lui apporter de nouveaux projets plus robustes et plus efficaces.\\ \\
Ainsi, \textbf{Talan Tunisie} a jugé utile d'opter pour l'architecture microservices de son ERP qui permet de pallier à certaines insuffisances de l'architecture monolithique, notamment la lourdeur de la maintenance et l'affectation du système par la défaillance d'une seule partie de ce dernier. Quelques modules ont déjà été migrés vers cette architecture (le module d'authentification, le module RH...). Notre travail a consisté à suivre toutes les étapes nécessaires (analyse préalable, conception, réalisation, intégration) pour assurer une telle migration et bénéficier des ses avantages.\\ \\ 
Signalons toutefois que l'intégration du microservice de gestion des candidats dans le service de découverte et dans le Gateway nous a demandé un effort particulier pour trouver le meilleur moyen de réaliser une telle opération qui constitue pour nous une tâche tout à fait nouvelle.\\ \\
Ce travail est certes perfectible. Certaines fonctionnalités pourront être intégrées ultérieurement dans le module de gestion des candidats. Citons en particulier : 
\begin{itemize}
    \item L'intégration d'un système de notification pour prévenir le recruteur de la planification d'un entretien, et informer le back-office de la validation de l'entretien par le recruteur.
    \item L'intégration d'une liste de questions qui peuvent être posées aux candidats par les recruteurs à chaque étapes de l'entretien.
    \item L'intégration d'un module de Machine Learning qui permet d'analyser les réponses des candidats. 
\end{itemize}
 


