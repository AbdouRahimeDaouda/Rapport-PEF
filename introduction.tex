\chapter*{Introduction générale}
\addcontentsline{toc}{chapter}{Introduction générale} 
\markboth{Introduction générale}{} 
De nos jours, l’évolution des technologies de l’information et de la communication a été accompagnée de profonds changements ayant affecté les processus métier au sein de l’entreprise qui doit dorénavant s'adapter rapidement aux nouvelles technologies pour assurer son évolution et sa pérennité.

En effet, l’équipe d'intégration doit suivre les évolutions technologiques qui ont apporté des changements dans les architectures logicielles. Ces changements ont pour objectif de développer des applications plus performantes, maintenables et extensibles. 

À cet égard, devant un développement exponentiel et continu de ses processus métier, tels que le processus de gestion des candidats qui est indispensable pour toute entreprise, il faut faire face à certains problèmes qui peuvent ralentir le fonctionnement de ses applications existantes tel que :  l’augmentation du temps de réponse, la redondance des modules et la difficulté de maintenance. 

C’est dans ce cadre que s’inscrit notre projet de fin d’études du cycle des ingénieurs à l’École Supérieure Privée de Technologies et d'Ingénierie(\textbf{TEK-UP}) réalisé à \textbf{Talan Tunisie Consulting}, société de services en ingénierie informatique. Notre tâche consiste à assurer une refonte architecturale du module de gestion des candidats de l’ERP Byblos de Talan, en passant de l’architecture monolithique vers une architecture à base de microservices. L'objectif étant d'avoir une structure robuste et fiable qui allège la complexité de l’architecture existante.

Le présent rapport décrit les différentes étapes de notre travail, et s’articule autour de quatre chapitres : 
\begin{itemize}
\item[-] Le premier chapitre comporte une brève présentation de l’organisme d’accueil et du cadre général de ce projet. Il expose aussi l’étude de l’existant et met l’accent sur la solution proposée et explique l’architecture microservices et la méthodologie de travail Scrum.
\item[-]Le deuxième chapitre présente, en premier lieu, une analyse détaillée des acteurs, des besoins fonctionnels et non fonctionnels du module de gestion des candidats de Byblos. En second lieu, il décrit le cas d’utilisation général ainsi que l’architecture globale de notre solution.
\item[-] Le troisième chapitre détaille la conception de ce module.  
\item[-] Le quatrième chapitre illustre l’intégration du micro-service tout en exposant les choix technologiques utilisés pour la réalisation de notre solution, ainsi que les résultats obtenus selon ces technologies. 
\item[-] Nous clôturons par une conclusion générale qui présente une récapitulation du travail réalisé et ouvre quelques perspectives.       
\end{itemize}